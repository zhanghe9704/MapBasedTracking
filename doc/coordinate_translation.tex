%% LyX 2.1.4 created this file.  For more info, see http://www.lyx.org/.
%% Do not edit unless you really know what you are doing.
\documentclass[english]{article}
\usepackage[T1]{fontenc}
\usepackage[latin9]{inputenc}
\usepackage{babel}
\begin{document}

\title{Phase space coordinate translation between PTC, MAD-X and COSY Infinity
9.1}

\maketitle
The coordinates in the 6D phase space of COSY Infinity 9.1 are defined
as follows. 

$r_{1}=x$,

$r_{2}=p_{x}/p_{0}$,

$r_{3}=y$,

$r_{4}=p_{y}/p_{0}$,

$r_{5}=l=-(t-t_{0})v_{0}\gamma/(1+\gamma)$,

$r_{6}=\delta_{k}=(K-K_{0})/K_{0}$,

where $p_{0}$, $K_{0}$, $v_{0}$, $t_{0}$ and $\gamma$ are the
momentum, kinetic energy, velocity, time of flight, and total energy
over $m_{0}c^{2}$, respectively, of the reference particle. The six
variables form three canoically conjungate pairs. 

The coordintes of MAD-X are defined as follows.

$X=x$,

$PX=p_{x}/p_{0}$,

$Y=y$,

$PY=p_{y}/p_{0}$,

$T=-c\cdot\delta t$,

$PT=\delta E/cp_{0}$,

where $\delta t$ and $\Delta E$ are the time difference and energy
difference with respect to the reference particle. $\delta t=t-t_{0}$
and $\delta E=K-K_{0}$. 

PTC uses the following coordinates: (PTC can use other coordinates.
But we assume it uses the following ones for our convenience.)

$X1=x$,

$X2=p_{x}/p_{0}$,

$X3=y$,

$X4=p_{y}/p_{0}$,

$X5=\delta E/cp_{0}$,

$X6=c\cdot\delta t$

Comparing the above three group of coordinates, the first four are
exactly the same. But the last two coordinates are defined defferently.
Using the relation $\beta=v_{0}/c=\sqrt{1-1/\gamma^{2}}$, one can
find the relation between $r_{5}$ and $T$. 
\[
r_{5}=-\delta t\frac{v_{0}\gamma}{1+\gamma}=\frac{T}{c}\frac{v_{0}\gamma}{1+\gamma}=T\frac{\beta\gamma}{1+\gamma}=T\sqrt{\frac{\gamma-1}{\gamma+1}}.
\]
Finding the relation between $r_{6}$ and $PT$ is also straightforward.
Using 
\[
K_{0}=cp_{0}\sqrt{\frac{\gamma-1}{\gamma+1}},
\]
We get
\[
r_{6}=\frac{\delta E}{K_{0}}=\frac{\delta E}{cp_{0}}\sqrt{\frac{\gamma+1}{\gamma-1}}=PT\sqrt{\frac{\gamma+1}{\gamma-1}}.
\]
The above relations allow us to convert from MAD-X coordinates to
COSY Infnity coordinates. It is easy to see the relation between PTC
coordinates and MAD-X coordinates is:
\[
X5=PT,\ \ \ X6=-T.
\]
And the relation between COSY Infinity 9.x is 
\[
r_{5}=-X6\sqrt{\frac{\gamma-1}{\gamma+1}},\ \ \ \ r_{6}=X5\sqrt{\frac{\gamma+1}{\gamma-1}}.
\]


It is good to see the translation is linear, which means the trancated
map using one group of coordinates can be converted into a map using
the other group of coordinates without introducing errors. 
\end{document}
